\section{Chapter 3}

We will make clear some asymptotic notation used in the text.
\begin{definition}
	\label{def:big-theta}
	Given functions $f,g:\mathbb{N}\to\mathbb{R}$ we say $f(n) = \Theta(g(n))$ if there exists $c_{1},c_{2}>0$ and $n_{0}\in\mathbb{N}$ such that $c_{2}\left|g(n)\right|\leq \left|f(n)\right|\leq c_{1}\left|g(n)\right|$. More generally, we consider $\Theta(g(n))$ to be the set of all functions that satisfy the above statement.
\end{definition}

\begin{definition}
	\label{def:big-O}
	Given functions $f,g:\mathbb{N}\to\mathbb{R}$ we say $f(n)=O(g(n))$ if there exists $c>0$ and $n_{0}\in\mathbb{N}$ such that $\left|f(n)\right|\leq c\left|g(n)\right|$ for all $n\geq n_{0}$. More generally, we consider $O(g(n))$ to be the set of all functions that satisfy the above statement.
\end{definition}

\begin{definition}
	\label{def:big-omega}
	Given functions $f,g:\mathbb{N}\to\mathbb{R}$ we say $f(n)=\Omega(g(n))$ if there exists $c>0$ and $n_{0}\in\mathbb{N}$ such that $c\left|g(n)\right|\leq\left|f(n)\right|$ for all $n\geq n_{0}$. More generally, we consider $\Omega(g(n))$ to be the set of all functions that satisfy the above statement.
\end{definition}

Notice that in all three of the above definitions we can consider the defined structure as a set or a identity between two functions. In this case we can write either $f\in O(g(n))$ or $f=O(g(n))$. Arguably the correct notation is $f\in O(g(n))$ as this works for the most general case where we consider $O(g(n))$ to be a set. However, we will often use the equality notation as a matter of simplicity since the function we compare with remains constant and we wish to make a statement about a particular element of the set in question.

We have the following two definitions which are more restrictive than those above.
\begin{definition}
	\label{def:little-oh}
	Given functions $f,g:\mathbb{N}\to\mathbb{R}$ we say $f(n)=o(g(n))$ if for every $c>0$ there exists $n_{0}\in\mathbb{N}$ such that $\left|f(n)\right|<c \left|g(n)\right|$ for all $n\geq n_{0}$. More generally, we consider the set $o(g(n))$ to be the set of all functions that satisfy the above statement.
\end{definition}

\begin{definition}
	\label{def:little-omega}
	Given functions $f,g:\mathbb{N}\to\mathbb{R}$ we say $f(n)=\omega(g(n))$ if for any positive constant $c>0$ there exists $n_{0}\in\mathbb{N}$ such that $c\left|g(n)\right|<\left|f(n)\right|$ for all $n\geq n_{0}$. More generally, we consider $\omega(g(n))$ to be the set of all functions that satisfy the above statement.
\end{definition}

\subsection{Asymptotic Notation - Exercises}

\subsubsection*{3.1--1}

Suppose $f,g:\mathbb{N}\to\mathbb{R}$ are asymptotically non-negative. Then there exists $n_{f}, n_{g}\in\mathbb{N}$ such that,
\begin{equation*}
	\begin{aligned}
		0 \leq f(n)& &\forall n\geq n_{f}\\
		0 \leq g(n)& &\forall n\geq n_{g}
	\end{aligned}
\end{equation*}
If we take $n_{0}:=\max(n_{f}, n_{g})$ then,
\begin{equation*}
	0\leq f(n), g(n) \quad \forall n\geq n_{0}
\end{equation*}
and so,
\begin{equation*}
	0 \leq \max(f(n), g(n)) \quad \forall n\geq n_{0}
\end{equation*}
By definition,
\begin{equation*}
	\max(f(n),g(n)) =
	\begin{cases}
		f(n) & \text{if }f(n)\geq g(n)\\
		g(n) & \text{otherwise}
	\end{cases}
\end{equation*}
so by the asymptotic non-negativity, $\max(f(n), g(n))\leq f(n) + g(n)$ and $\max(f(n), g(n))\geq \frac{f(n)+g(n)}{2}$ for all $n\geq n_{0}$. Hence, $\max(f(n), g(n))\in\Theta(f(n)+g(n))$.

\subsubsection*{3.1--2}

Let $a,b\in\mathbb{R}$ such that $b>0$. Choose $n_{0}\in\mathbb{N}$ such that $n_{0}\geq\lceil|a|\rceil$ then,
\begin{equation*}
	(n+a)^{b}\leq 2^{b}n^{b} \quad \forall n\geq n_{0}
\end{equation*}
so $(n+a)^{b}=O(n^{b})$. Working in reverse,
\begin{equation*}
	\begin{aligned}
		(n+a)^{b} &\geq cn^{b}\\
		n+a &\geq c^{\frac{1}{b}}n\\
		n-c^{\frac{1}{b}}n &\geq -a\\
		n &\geq \frac{-a}{1-c^{\frac{1}{b}}}
	\end{aligned}
\end{equation*}
it then suffices to choose $c<1$. So for any $c=\frac{1}{2}$ and $n_{0}\geq \frac{-a}{1-c^{\frac{1}{b}}}$ we then have,
\begin{equation*}
	(n+a)^{b}\geq cn^{b}
\end{equation*}
hence, $(n+a)^{b}\in\Omega(n^{b})$. Therefore, $(n+a)^{b}\in\Theta(n^{b})$.

\subsubsection*{3.1--3}

The statement ``The running time of the algorithm $A$ is at least $O(n^{2})$'' is meaningless as $O(n^{2})$ gives an asymptotic upper bound, or in terms of algorithms, the worst-case running time. Since $f(n)=O(n^{2})$ means there exists $c>0$ and $n_{0}\in\mathbb{N}$ such that,
\begin{equation*}
	0\leq f(n)\leq cn^{2}\quad\forall n\geq n_{0}
\end{equation*}
It is clear from the definition that the algorithm runtime may or may not be less than $cn^{2}$.

\subsubsection*{3.1--4}

\begin{itemize}
	\item [(a)]
		Given $c\geq 2$ we can write $2^{n+1}=2\cdot2^{n}\leq c2^{n}$ for any $n\in\mathbb{N}$. So $2^{n+1}=O(2^{n})$.
	\item [(b)]
		Suppose there exists $c>0$ and $n_{0}\in\mathbb{N}$ such that $0\leq 2^{2n}\leq c2^{n}$ for all $n\geq n_{0}$.Then $0\leq 2^{n} \leq c$ for all $n\geq n_{0}$ which is clearly a contradiction. Hence, $2^{2n}\neq O(2^{n})$.
\end{itemize}

\subsubsection*{3.1--5}

\begin{theorem}
	\label{thm:big-theta}
	For any two functions $f,g:\mathbb{N}\to\mathbb{R}$ we have $f(n)=\Theta(g(n))$ if and only if $f(n)=O(g(n))$ and $f(n)=\Omega(g(n))$.
\end{theorem}

\begin{proof}
	First suppose $f(n)=\Theta(g(n))$, then by definition there exists $c_{1},c_{2}>0$ and $n_{0}\in\mathbb{N}$ such that,
	\begin{equation*}
		c_{2}\left|g(n)\right| \leq \left|f(n)\right| \leq c_{1}\left|g(n)\right| \quad\forall n\geq n_{0}
	\end{equation*}
	and so $f(n)$ satisfies the conditions of $O(g(n))$ and $\Omega(g(n))$.

	Now consider $f(n)=O(g(n))$ and $f(n)=\Omega(g(n))$. Then, there exists $c_{1},c_{2}>0$ and $n_{1},n_{2}\in\mathbb{N}$ such that,
	\begin{equation*}
		\begin{aligned}
			\left|f(n)\right| &\leq c_{1}\left|g(n)\right|& &\forall n\geq n_{1}\\
			c_{2}\left|g(n)\right| &\leq \left|f(n)\right|& &\forall n\geq n_{2}
		\end{aligned}
	\end{equation*}
	Set $n_{0}:=\max(n_{1},n_{2})$ then we have,
	\begin{equation*}
		c_{2}\left|g(n)\right| \leq \left|f(n)\right| \leq c_{1}\left|g(n)\right|\quad\forall n\geq n_{0}
	\end{equation*}
	and so $f(n)=\Theta(g(n))$.
\end{proof}

\subsubsection*{3.1--6}

\begin{corollary}
	\label{cor:big-theta}
	The running time of an algorithm is $\Theta(g(n))$ if and only if its worst-case running time is $O(g(n))$ and its best-case running time is $\Omega(g(n))$.
\end{corollary}

\begin{proof}
	This is an immediate application of Theorem \ref{thm:big-theta}. Let $W(n)$ denote the worst-case running time, $B(n)$ denote the best-case running time and $T(n)$ denote the running time of the algorithm. Then,
	\begin{equation*}
		B(n) \leq T(n) \leq W(n)\quad\forall n\in\mathbb{N}
	\end{equation*}
	and so $T(n)=O(g(n))$ and $T(n)=\Omega(g(n))$. Hence, by Theorem \ref{thm:big-theta} $T(n)=\Theta(g(n))$.
\end{proof}

\subsubsection*{3.1--7}

\begin{proposition}
	\label{prop:little-intersection}
	The set $o(g(n))\cap\omega(g(n))$ is empty.
\end{proposition}

\begin{proof}
	It is sufficient to show that for any $f(n)=o(g(n))$ we have $f(n)\neq \omega(g(n))$. Suppose $f(n)=o(g(n))$ then for every $c>0$ there exists $n_{0}\in\mathbb{N}$ such that $\left|f(n)\right| < c\left|g(n)\right|$ for all $n\geq n_{0}$. Assume $f(n)=\omega(g(n))$ then for any $c>0$ we can choose $n_{0}\in\mathbb{N}$ so that,
	\begin{equation*}
		c\left|g(n)\right| < \left|f(n)\right| < c\left|g(n)\right|
	\end{equation*}
	This is clearly a contradiction and so we have $f(n)\neq \omega(g(n))$. Hence, $o(g(n))\cap\omega(g(n))=\emptyset$.
\end{proof}

\subsubsection*{3.1--8}

\begin{equation*}
	\begin{aligned}
		O(g(n,m)) := \{f(n,m)\colon&\exists c>0 \text{ and } \exists n_{0},m_{0}\in\mathbb{N} \text{ such that }\\
		&\left|f(n,m)\right| \leq c\left|g(n,m)\right|\,\forall n\geq n_{0}\text{ or }m\geq m_{0}\}\\
		\Omega(g(n,m)) := \{f(n,m)\colon&\exists c>0 \text{ and } \exists n_{0},m_{0}\in\mathbb{N} \text{ such that }\\
		&c\left|g(n,m)\right| \leq \left|f(n,m)\right|\,\forall n\geq n_{0}\text{ or }m\geq m_{0}\}\\
		\Theta(g(n,m)) := \{f(n,m)\colon&\exists c_{1},c_{2}>0 \text{ and } \exists n_{0},m_{0}\in\mathbb{N} \text{ such that }\\
		&c_{2}\left|g(n,m)\right|\leq\left|f(n,m)\right| \leq c_{1}\left|g(n,m)\right|\,\forall n\geq n_{0}\text{ or }m\geq m_{0}\}
	\end{aligned}
\end{equation*}

\subsection{Standard Notations and Common Functions}

\subsubsection*{3.2--1}

\begin{proposition}
	\label{prop:monotone}
	Let $f,g:\mathbb{N}\to\mathbb{N}$ such that $f$ and $g$ are monotonically increasing. Then $f(n)+g(n)$, $f(g(n))$ and $g(f(n))$ are monotonically increasing. Moreover, if $f$ and $g$ are non-negative then $f(n)\cdot g(n)$ is monotonically increasing.
\end{proposition}

\begin{proof}
	Clearly,
	\begin{equation*}
		f(n)+g(n)\leq f(m)+g(m)
	\end{equation*}
	for all $n,m\in\mathbb{N}$ such that $n\leq m$. Now if $n,m\in\mathbb{N}$ so that $n\leq m$ then,
	\begin{equation*}
		g(n)\leq g(m)\implies f(g(n))\leq f(g(m))
	\end{equation*}
	Likewise the result will be true for $g(f(n))$. If $f$ and $g$ are non-negative then $f(n)\cdot g(n)$ is non-negative for all $n\in\mathbb{N}$. So for $n,m\in\mathbb{N}$ such that $n\leq m$,
	\begin{equation*}
		f(n)\cdot g(n)\leq f(m)\cdot g(n)\leq f(m)\cdot g(m)
	\end{equation*}
\end{proof}

\begin{remark}
	\label{rem:monotone}
	The results of Proposition \ref{prop:monotone} are strict if $f$ and $g$ are strictly monotonically increasing.
\end{remark}

\subsubsection*{3.2--2}

We write,
\begin{equation*}
	a^{\log_{b}c} = a^{\frac{\log_{a}c}{\log_{a}b}} = \left(a^{\log_{a}c}\right)^{\frac{1}{\log_{a}b}} = c^{\frac{1}{\log_{a}b}}
\end{equation*}
Now,
\begin{equation*}
	\frac{1}{\log_{a}b}=\frac{\log_{b}a}{\log_{b}b}=\log_{b}a
\end{equation*}
Hence, from above,
\begin{equation*}
	a^{\log_{b}c} = c^{\frac{1}{\log_{a}b}}=c^{\log_{b}a}
\end{equation*}

\subsubsection*{3.2--3}

Stating Sterling's approximation,
\begin{equation*}
	n! = \sqrt{2\pi n}\left(\frac{n}{e}\right)^{n}\left(1+\Theta\left(\frac{1}{n}\right)\right)
\end{equation*}
we then have,
\begin{equation*}
	\begin{aligned}
		\lg n! &= \lg\left(\sqrt{2\pi n}\left(\frac{n}{e}\right)^{n}\left(1+\Theta\left(\frac{1}{n}\right)\right)\right)\\
		&= \lg \sqrt{2\pi n} + n\lg \frac{n}{e} + \lg\left(1+\Theta\left(\frac{1}{n}\right)\right)\\
		&= \frac{1}{2}+\frac{1}{2}\lg\pi + \frac{1}{2}\lg n + n\lg\frac{1}{e} + n\lg n + \lg\left(1+\Theta\left(\frac{1}{n}\right)\right)\\
		&= \Theta(1) + \Theta(1) + \Theta(\lg n) + \Theta(n) + \Theta(n\lg n) + \lg\Theta(1) = \Theta(n\lg n)
	\end{aligned}
\end{equation*}

Now,
\begin{equation*}
	n! = n\cdot(n-1)\cdots2\cdot1 < n\cdot n\cdots n\cdot n = n^{n}\quad\forall n\in\mathbb{N}
\end{equation*}
So for $c\geq 1$ we have the desired property. For $0<c<1$ choose $n_{0}$ so that $cn_{0} > 1$ then,
\begin{equation*}
	n! = n\cdot(n-1)\cdots2\cdot1 < n\cdot(n-1)\cdots2\cdot cn < c n^{n}\quad\forall n\geq n_{0}
\end{equation*}
Hence, $n!=o(n^{n})$.

Finally,
\begin{equation*}
	n! = n\cdot(n-1)\cdots 2\cdot1 = n\cdot(n-1)\cdots 2 > 2^{n-1}=\frac{1}{2}2^{n}\quad\forall n\in\mathbb{N}
\end{equation*}
So for $c\leq\frac{1}{2}$ we have the desired property. For $c>\frac{1}{2}$ choose $n_{0}$ so that $n_{0}>c2^{2}$ so,
\begin{equation*}
	n! = n\cdot(n-1)\cdots2>c2^{2}\cdot(n-1)\cdots2>c2^{n}\quad\forall n\geq n_{0}
\end{equation*}
Hence $n!=\Omega(2^{n})$.

\subsubsection*{3.2--4}

Assume $\lceil\lg n\rceil!=O(n^{m})$ for some $m\in\mathbb{N}$. Then there exists $c>0$ and $n_{0}\in\mathbb{N}$ such that $0\leq \lceil\lg n\rceil!\leq cn^{m}$ for all $n\geq n_{0}$. From exercise 3.2--3 we have $n!=\omega(2^{n})$ so for any fixed $n\geq n_{0}$ and for any $\varepsilon>c(2^{n})^{m}$ there exists $N_{0}\geq n_{0}$ such that,
\begin{equation*}
	\begin{aligned}
		c(2^{n})^{m}2^{n} < \varepsilon 2^{n} < n! = \lceil\lg 2^{n}\rceil! \leq c (2^{n})^{m}\quad\forall n\geq N_{0}
	\end{aligned}
\end{equation*}
A contradiction and so $\lceil\lg n\rceil!\neq O(n^{m})$.

For any $n\in\mathbb{N}$ such that $n\geq 2$ take $k=\lg\lg n$ so that we can write $n=2^{2^{k}}$ then $\lceil\lg\lg n\rceil!\leq n^{2}$, for some $m\in\mathbb{N}$, is equivalent to $\lceil k\rceil!\leq 2^{2^{k^{2}}}$. Setting $m=\lceil k\rceil$ we have
\begin{equation*}
	\begin{aligned}
		\lceil k\rceil! = m! = m\cdot(m-1)\cdots2\cdot1 &\leq 2^{m}\cdot2^{m}\cdots2^{m}\cdot2^{m}\\
		&\leq 2^{m^{2}} \leq 2^{2^{m^{2}}} \leq 2^{2^{k^{2}}}
	\end{aligned}
\end{equation*}
Hence, $\lceil\lg\lg n\rceil!=O(n^{2})$.

\subsubsection*{3.2--5}

By definition $\lg^{*}n:=\min\left\{i\geq 0\colon \lg^{(i)}n\leq 1\right\}$ where,
\begin{equation*}
	f^{(i)}(n) =
	\begin{cases}
		n& \text{if }i=0\\
		f(f^{(i)}(n))& \text{if }i>0
	\end{cases}
\end{equation*}
For any $n\geq 1$ take $k=\lg n$ then $n=2^{k}$ and so $\lg^{*}2^{k} = 1 + \lg^{*}k$ and $\lg^{*}(\lg 2^{k})=\lg^{*}k$. Then,
\begin{equation*}
	\frac{\lg(\lg^{*}2^{k})}{\lg^{*}(\lg 2^{k})} = \frac{\lg(1 + \lg^{*}k)}{\lg^{*}k}
\end{equation*}
By an application of L'hopital's rule we can consider the quotient above to be a subsequence of the real valued quotient, in which case,
\begin{equation*}
	\lim_{k\to\infty}\frac{\lg(1+\lg^{*}k)}{\lg^{*}k} = \lim_{k\to\infty}\frac{1}{1 + \lg^{*}k}\to 0
\end{equation*}
So $\lg^{*}(\lg n)$ is asymptotically larger.

\subsubsection*{3.2--6}

We have,
\begin{equation*}
	\varphi = \frac{1+\sqrt{5}}{2}\qquad \hat{\varphi} = \frac{1-\sqrt{5}}{2}
\end{equation*}
Then,
\begin{equation*}
	\varphi^{2} = \frac{6+2\sqrt{5}}{4} = \frac{3 + \sqrt{5}}{2} = \frac{1 + \sqrt{5} + 2}{2} = \varphi + 1
\end{equation*}
and,
\begin{equation*}
	\hat{\varphi}^{2} = \frac{6 - 2\sqrt{5}}{2} = \frac{3 - \sqrt{5}}{2} = \frac{1 - \sqrt{5} + 2}{2} = \hat{\varphi} + 1
\end{equation*}

\subsubsection*{3.2--7}

Suppose $n=0$ then,
\begin{equation*}
	\frac{\varphi^{0} - \hat{\varphi}^{0}}{\sqrt{5}} = \frac{0}{\sqrt{5}} = F_{0}
\end{equation*}
If $n=1$,
\begin{equation*}
	\frac{\varphi - \hat{\varphi}}{\sqrt{5}} = \frac{\frac{2\sqrt{5}}{2}}{\sqrt{5}} = 1 = F_{1}
\end{equation*}
Assume that for some $n$ we have,
\begin{equation*}
	F_{n} = \frac{\varphi^{n} - \hat{\varphi}^{n}}{\sqrt{5}}
\end{equation*}
then,
\begin{equation*}
	\begin{aligned}
		F_{n+1} = F_{n} + F_{n-1} &= \frac{\varphi^{n} - \hat{\varphi}^{n}}{\sqrt{5}} + \frac{\varphi^{n-1} - \hat{\varphi}^{n-1}}{\sqrt{5}}\\
		&= \frac{\varphi^{n} + \varphi^{n-1} - \hat{\varphi}^{n} - \hat{\varphi}^{n-1}}{\sqrt{5}}\\
		&= \frac{\varphi^{n-1}(\varphi + 1) - \hat{\varphi}^{n-1}(\hat{\varphi}+1)}{\sqrt{5}}\\
		&= \frac{\varphi^{n-1}\varphi^{2} - \hat{\varphi}^{n-1}\hat{\varphi}^{2}}{\sqrt{5}}\\
		&= \frac{\varphi^{n+1} - \hat{\varphi}^{n+1}}{\sqrt{5}}
	\end{aligned}
\end{equation*}

\subsubsection*{3.2--8}

If $k\lg k = \Theta(n)$ then there exists $c_{1},c_{2}>0$ and $n_{0},k_{0}\in\mathbb{N}$ such that,
\begin{equation*}
	0\leq c_{2}n\leq k\lg k \leq c_{1}n\quad\forall n\geq n_{0}, \forall k\geq k_{0}
\end{equation*}
which gives,
\begin{equation*}
	0 \leq c_{2}\frac{n}{\lg k}\leq k \leq c_{1}\frac{n}{\lg k}\quad\forall n\geq n_{0}, \forall k\geq k_{0}
\end{equation*}
From $k\geq k_{0}$ we have,
\begin{equation*}
	0 \leq k \leq \frac{c_{1}}{\lg k_{0}}n\quad\forall n\geq n_{0}, \forall k\geq k_{0}
\end{equation*}
so $k = O(n)$ and by symmetry $n = \Omega(k)$.
