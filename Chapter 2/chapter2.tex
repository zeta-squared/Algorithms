\section{Chapter 2}
\label{sec:chp2}

\section*{2.1--2}

\begin{algorithmic}[1]
	\Procedure{\textit{INSERTION SORT}}{$A$}
	\For{$i = 1$ \textbf{to} $A.\text{length} - 1$}
	\State $\text{key} = A[i]$
	\State $j = i - 1$
	\While{$j > 0$ \textbf{and} $A[j] < \text{key}$}
	\State $A[j+1] = A[j]$
	\State $j = j - 1$
	\EndWhile
	\State $A[i+1]=\text{key}$
	\EndFor
	\EndProcedure
\end{algorithmic}

\subsection*{2.1--3}

\begin{algorithmic}[1]
	\Procedure{\textit{LINEAR SEARCH}}{$A$, $\nu$}
	\For{$i = 0$ \textbf{to} $A.\text{length} - 1$}
	\If{$A[i] == \nu$}
	\State \Return $i$
	\EndIf
	\State \Return NIL
	\EndFor
	\EndProcedure
\end{algorithmic}

At the start of each iteration of the \textbf{for} loop (lines 2--7) $i-1$ is not an index of $A$ such that $A[i-1]=\nu$.

Let us now prove the correctness of our algorithm. Suppose $i=0$, then $i-1$ is clearly not an index of $A$ and hence $A[i-1]$ is undefined. Now suppose the loop invariant is true for some $i$, that is, $i-1$ is not an index of $A$ such that $A[i-1]=\nu$, or equivalently, $A[i-1]\neq\nu$. Then at line 3 the \textbf{if} loop will \textbf{return} $i$ if $A[i]=\nu$, in which case the \textbf{for} loop terminates and there is no further iteration. Otherwise, if $A[i]\neq\nu$ then at the start of the next for loop iteration $(i+1)-1$ is not an index of $A$ such that $A[(i+1)-1]=\nu$. Finally, for termination to occur we have either $i=n+1$ where $n=A.length$ in which case the algorithm returns NIL indicating $\nu$ is not an element of $A$. Otherwise, termination occurs because of the nested \textbf{if} on line 3 which causes the algorithm to return $i$ which indicates the index of $A$ such that $A[i]=\nu$.

\subsection*{2.1--4}

\noindent\textbf{Input:} Two sequences of $n$ integers, $A=(a_{1},\ldots,a_{n})$ and $B=(b_{1},\ldots,b_{n})$, such that $0\leq a_{i},b_{i}\leq 1$ for $i=1,\ldots,n$. Least significant digits are first.\\

\noindent\textbf{Output:} An array $C=(c_{1},\ldots,c_{n},c_{n+1})$ such that $0\leq c_{i}\leq 1$ for $i=1,\ldots,n+1$ and $C'=A'+B'$ where $\cdot'$ is the integer represented by $\cdot$.\\

\begin{algorithmic}[1]
	\Procedure{\textit{BINARY ADDITION}}{$A$, $B$}
	\State \textbf{Define} integer $C[A.\text{length}+1]$
	\State $\text{overflow}=0$
	\For{$i=0$ \textbf{to} $A.\text{length}-1$}
	\State $C[i] = (A[i]+B[i]+\text{overflow})\text{ \% }2$
	\State $\text{overflow} = (A[i]+B[i]+\text{overflow})/2$
	\EndFor
	\State $C[i]=\text{overflow}$
	\State \Return C
	\EndProcedure
\end{algorithmic}
