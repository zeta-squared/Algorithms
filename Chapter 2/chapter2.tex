\section{Chapter 2}
\label{sec:chp2}

\section*{2.1--2}

\begin{codebox}
	\Procname{$\proc{Decreasing-Insertion-Sort}(A)$}
	\li \For $i = 1$ \To $A.\text{length}-1$
	\li	\Do
				key $= A[i]$
	\li 	$j = i - 1$
	\li		\While $j > 0$ \textbf{and} $A[j] <$ key
	\li		\Do
					$A[j+1] = A[j]$
	\li			$j = j - 1$
				\End
	\li		$A[i+1] =$ key
			\End
\end{codebox}

\subsection*{2.1--3}

\begin{codebox}
	\Procname{$\proc{Linear-Search}(A,\nu)$}
	\li \For $i = 0$ \To $A.\text{length}-1$
	\li \Do
				\If $A[i] \isequal \nu$
	\li		\Then
					\Return $i$
				\End
			\End
	\li		\Return \const{nil}
\end{codebox}

\begin{invariant}
	At the start of each iteration of the \textbf{for} loop (lines 1--4) $i-1$ is not an index of $A$ such that $A[i-1]=\nu$.
\end{invariant}

\begin{proof}
	Let us now prove the correctness of our algorithm. Suppose $i=0$, then $i-1$ is clearly not an index of $A$ and hence $A[i-1]$ is undefined. Now suppose the loop invariant is true for some $i$, that is, $i-1$ is not an index of $A$ such that $A[i-1]=\nu$, or equivalently, $A[i-1]\neq\nu$. Then at line 3 the \textbf{if} loop will \textbf{return} $i$ if $A[i]=\nu$, in which case the \textbf{for} loop terminates and there is no further iteration. Otherwise, if $A[i]\neq\nu$ then at the start of the next for loop iteration $(i+1)-1$ is not an index of $A$ such that $A[(i+1)-1]=\nu$. Finally, for termination to occur we have either $i=n+1$ where $n=A.\text{length}$ in which case the algorithm returns NIL indicating $\nu$ is not an element of $A$. Otherwise, termination occurs because of the nested \textbf{if} on line 3 which causes the algorithm to return $i$ which indicates the index of $A$ such that $A[i]=\nu$.
\end{proof}

\subsection*{2.1--4}

\begin{inp}
	Two sequences of $n$ integers, $A=(a_{1},\ldots,a_{n})$ and $B=(b_{1},\ldots,b_{n})$, such that $0\leq a_{i},b_{i}\leq 1$ for $i=1,\ldots,n$. Least significant digits are first.
\end{inp}

\begin{outp}
	An array $C=(c_{1},\ldots,c_{n},c_{n+1})$ such that $0\leq c_{i}\leq 1$ for $i=1,\ldots,n+1$ and $C'=A'+B'$ where $\cdot'$ is the integer represented by $\cdot$.
\end{outp}

\begin{codebox}
	\Procname{$\proc{Binary-Addition}(A,B)$}
	\li \Define integer $C[A.\text{length}+1]$
	\li overflow $=0$
	\li	\For $i = 0$ \To $A.\text{length}-1$
	\li \Do
				$C[i] = (A[i] + B[i] + \text{overflow})$ \% $2$
	\li		overflow $= (A[i] + B[i] + \text{overflow})/2$
			\End
	\li $C[i] =$ overflow
	\li \Return $C$
\end{codebox}

\subsection*{2.2--1}

The function is $O(n^{3})$

\subsection*{2.2--2}

\begin{multicols}{2}
	\begin{codebox}
		\Procname{$\proc{Selection-Sort}(A)$}
		\li	\For $i = 0$ \To $A.\text{length}-2$
		\li	\Do
					min $= i$
		\li		\For $j = i+1$ \To $A.$length$-1$
		\li		\Do
						\If $A[j] < A[\text{min}]$
		\li			\Then
							min = $j$
						\End
					\End
		\li $M = A[\text{min}]$
		\li $A[\text{min}] = A[i]$
		\li $A[i] = M$
					\End
		\end{codebox}
		\columnbreak
		\begin{tabular}{ l l }
			cost & times\\[4pt]
			$c_{1}$ & $n$\\
			$c_{2}$ & $n-1$\\
			$c_{3}$ & $\sum_{i=0}^{n}(n-i+1)$\\
			$c_{4}$ & $\sum_{i=0}^{n}(n-i)$\\
			$c_{5}$ & $\sum_{i=0}^{n}t_{i}$\\
			$c_{6}$ & $n-1$\\
			$c_{7}$ & $n-1$\\
			$c_{8}$ & $n-1$
		\end{tabular}
\end{multicols}

\begin{invariant}
	At the start of each iteration of the \textbf{for} loop (lines 1--8) the sub-array $A[0\ldots i]$ is sorted in non-decreasing order.
\end{invariant}

The algorithm only needs to run for the first $n-1$ elements since this will arrange the $n-1$ smallest elements in non-decreasing order, ensuring the $n^{\text{th}}$ element at the end is in the appropriate position. That is, $A[n]\geq A[i]$ for $i=0,\ldots,n-2$.

The best-case running time occurs when the given array is already sorted from smallest to largest. In such a case $t_{i}=0$ since we never need to re-assign the minimum index. The runtime equation is, 
\begin{equation*}
	\begin{aligned}
		T(n) &= c_{1}n + (c_{2}+c_{6}+c_{7}+c_{8})(n-1) + c_{3}\sum_{i=0}^{n}(n-i+1) + c_{4}\sum_{i=0}^{n}(n-i)\\
		&= c_{1}n + (c_{2}+c_{6}+c_{7}+c_{8})(n-1) + c_{3}\left((n+1) + \frac{n}{2}(n+1)\right) + c_{4}\left(n + \frac{n}{2}(n-1)\right)\\
		&= (c_{3}+c_{4})\frac{n^{2}}{2} + (c_{1} + c_{2} + c_{6} + c_{7} + c_{8} + \frac{3}{2}c_{3} + \frac{1}{2}c_{4})n + (c_{2}+c_{6}+c_{7}+c_{8} + c_{3})
	\end{aligned}
\end{equation*}
and so the best-case running time is $O(n^{2})$. In a worst-case scenario, the array given to the procedure is in descending order, however this would only include an additional term to $T(n)$ above,
\begin{equation*}
	c_{5}\sum_{i=0}^{n}(n-1) = c_{5}\left(n + \frac{n}{2}(n-1)\right) = \frac{1}{2}c_{5}(n^{2}+n)
\end{equation*}
since here line 5 will re-assign the minimum for all remaining entries in the array. So the runtime in a worst-case scenario is also $O(n^{2})$.

\subsection*{2.2--3}

\begin{multicols}{2}
	\begin{codebox}
		\Procname{$\proc{Linear-Search}(A,\nu)$}
		\li \For $i = 0$ \To $A.\text{length}-1$
		\li	\Do
					\If $A[i] \isequal \nu$
		\li		\Then
						\Return $i$
					\End
				\End
		\li \Return \const{nil}
	\end{codebox}
	\columnbreak
	\begin{tabular}{ l l }
		cost & times\\
		$c_{1}$ & $n+1$\\
		$c_{2}$ & $n$\\
		$c_{3}$ & $t_{1}$\\
		$c_{4}$ & $t_{2}$
	\end{tabular}
\end{multicols}

If each of the $n$ elements of $A$ have equal probability $p$ to be $\nu$ then the expected value is,
\begin{equation*}
	E[\nu] = 0\times\frac{1}{n}+1\times\frac{1}{n} + 2\times\frac{1}{n} + \cdots + n\times\frac{1}{n} = \frac{1}{n}\sum_{i=1}^{n}i = \frac{1}{n}\frac{n}{2}(n+1) = \frac{n+1}{2}
\end{equation*}
and hence on average we need to search through $\frac{n+1}{2}$ elements to find $\nu$. In the worst case we need to search $n$ elements since $\nu$ is not present in $A$. We have the following runtime equation,
\begin{equation*}
	T(n) = c_{1}(n+1) + c_{2}n + c_{3}t_{1} + c_{4}t_{2}
\end{equation*}
In the average-case $t_{1}=\frac{1}{2}=t_{2}$ then,
\begin{equation*}
	T(n) = (c_{1} + c_{2})n + c_{1} + \frac{1}{2}(c_{3} + c_{4})
\end{equation*}
and so the runtime is $O(n)$. In the worst-case $t_{1}=0$ and $t_{2}=1$ so the runtime equation is,
\begin{equation*}
	T(n) = (c_{1} + c_{2})n + c_{1} + c_{3}
\end{equation*}
and so we still have $O(n)$ runtime.

\subsection*{2.2--4}

Implement a checking loop/statement to return the procedure if in a best-case scenario. For example in Selection-Sort we can implement an initial loop that checks if the given array is already in sorted order and then return,
\begin{multicols}{2}
	\begin{codebox}
		\Procname{}
		\li \For $i = 0$ \To $A.\text{length}-2$
		\li \Do
					\If $A[i] > A[i+1]$
		\li		\Then
						\Break
					\End
				\End
		\li \If $i \isequal A.\text{length}-2$
		\li \Then
					\Return
				\End
	\end{codebox}
	\begin{tabular}{ l l }
		cost & times\\
		$c_{1}$ & $n$\\
		$c_{2}$ & $n-1$\\
		$c_{3}$ & $t_{1}$\\
		$c_{4}$ & $1$\\
		$c_{5}$ & $t_{2}$
	\end{tabular}
\end{multicols}
In such a case the runtime will be,
\begin{equation*}
	T(n) = (c_{1}+c_{2})n - c_{2} + c_{4} + c_{5}
\end{equation*}
which is $O(n)$ a significant improvement over $O(n^{2})$ in the above exercise.
