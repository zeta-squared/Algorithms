\section{Chapter 2}
\label{sec:chp2}

\section*{2.1--2}

\begin{algorithmic}[1]
	\Procedure{\textit{INSERTION SORT}}{$A$}
	\For{$i = 1$ \textbf{to} $A.\text{length} - 1$}
	\State $\text{key} = A[i]$
	\State $j = i - 1$
	\While{$j > 0$ \textbf{and} $A[j] < \text{key}$}
	\State $A[j+1] = A[j]$
	\State $j = j - 1$
	\EndWhile
	\State $A[i+1]=\text{key}$
	\EndFor
	\EndProcedure
\end{algorithmic}

\subsection*{2.1--3}

\begin{algorithmic}[1]
	\Procedure{\textit{LINEAR SEARCH}}{$A$, $\nu$}
	\For{$i = 0$ \textbf{to} $A.\text{length} - 1$}
	\If{$A[i] == \nu$}
	\State \Return $i$
	\EndIf
	\State \Return NIL
	\EndFor
	\EndProcedure
\end{algorithmic}

\begin{invariant}
	At the start of each iteration of the \textbf{for} loop (lines 2--7) $i-1$ is not an index of $A$ such that $A[i-1]=\nu$.
\end{invariant}

\begin{proof}
	Let us now prove the correctness of our algorithm. Suppose $i=0$, then $i-1$ is clearly not an index of $A$ and hence $A[i-1]$ is undefined. Now suppose the loop invariant is true for some $i$, that is, $i-1$ is not an index of $A$ such that $A[i-1]=\nu$, or equivalently, $A[i-1]\neq\nu$. Then at line 3 the \textbf{if} loop will \textbf{return} $i$ if $A[i]=\nu$, in which case the \textbf{for} loop terminates and there is no further iteration. Otherwise, if $A[i]\neq\nu$ then at the start of the next for loop iteration $(i+1)-1$ is not an index of $A$ such that $A[(i+1)-1]=\nu$. Finally, for termination to occur we have either $i=n+1$ where $n=A.\text{length}$ in which case the algorithm returns NIL indicating $\nu$ is not an element of $A$. Otherwise, termination occurs because of the nested \textbf{if} on line 3 which causes the algorithm to return $i$ which indicates the index of $A$ such that $A[i]=\nu$.
\end{proof}

\subsection*{2.1--4}

\begin{inp}
	Two sequences of $n$ integers, $A=(a_{1},\ldots,a_{n})$ and $B=(b_{1},\ldots,b_{n})$, such that $0\leq a_{i},b_{i}\leq 1$ for $i=1,\ldots,n$. Least significant digits are first.
\end{inp}

\begin{outp}
	An array $C=(c_{1},\ldots,c_{n},c_{n+1})$ such that $0\leq c_{i}\leq 1$ for $i=1,\ldots,n+1$ and $C'=A'+B'$ where $\cdot'$ is the integer represented by $\cdot$.
\end{outp}

\begin{algorithmic}[1]
	\Procedure{\textit{BINARY ADDITION}}{$A$, $B$}
	\State \textbf{Define} integer $C[A.\text{length}+1]$
	\State $\text{overflow}=0$
	\For{$i=0$ \textbf{to} $A.\text{length}-1$}
	\State $C[i] = (A[i]+B[i]+\text{overflow})\text{ \% }2$
	\State $\text{overflow} = (A[i]+B[i]+\text{overflow})/2$
	\EndFor
	\State $C[i]=\text{overflow}$
	\State \Return C
	\EndProcedure
\end{algorithmic}

\subsection*{2.2--1}

The function is $O(x^{3})$

\subsection*{2.2--2}

\begin{algorithmic}[1]
	\Procedure{\textit{SELECTION SORT}}{$A$}
	\For{$i=0$ \textbf{to} $A.\text{length}-2$}
	\State $\text{min} = i$
	\For{$j=i+1$ \textbf{to} $A.\text{length}-1$}
	\If{$A[j] < A[\text{min}]$}
	\State $\text{min} = j$
	\EndIf
	\EndFor
	\State $M=A[\text{min}]$
	\State $A[\text{min}] = A[i]$
	\State $A[i] = M$
	\EndFor
	\EndProcedure
\end{algorithmic}

\begin{invariant}
	At the start of each iteration of the \textbf{for} loop (lines 2--12) the sub--array $A[0\ldots i]$ is sorted in non--decreasing order.
\end{invariant}

The algorithm only needs to run for the first $n-1$ elements since this will arrange the $n-1$ smallest elements in non--decreasing order, ensuring the $n^{\text{th}}$ element at the end is in the appropriate position. That is, $A[n]\geq A[i]$ for $i=0,\ldots,n-2$.
