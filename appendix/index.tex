\subsection*{Resource Models}

When we consider a model for analysing the time complexity of an algorithm, such as the
\textbf{\textit{random-access machine (RAM)}} model we need to define a word size of data. Here \textit{word}
indicates some object to be stored in data. This topic is important because it creates a limit on how much
information can be stored in a single word. If we do not make such assumptions then arguably one can store an
infinite amount of data in each word and so every algorithm has constant runtime. Clearly this cannot be true.
For our purposes when we want to work with inputs of size $n$ we need to be able to index up to $n$.
This leads us to require that a word of data must be able to store the numerical value $n$. This is
essentially determining how many bits we require in our machine to be able to store a word with numerical
values up to, and including, $n$.

This representation in bits can be seen as follows. Since $\lg$ is base $2$ we then have $2^{\lg n}=n$.
However note that in machine counting we start at $0$ so if we have $\lg n$ bits in a machine we can count
from $0$ to $n-1$ (equivalently $1$ to $n$). Remember that $n$ is arbitrary, however it cannot be varied
after we set it in our model. So while our machine may recieve data made up of more than one words, for
instance $n^{c}$ in size, we can still index this data in constant time since $n^{c} = 2^{c\lg n}$. Meaning
that $c\lg n$ operations can be performed in constant time.

The importance of this calculation becomes more obvious when we deal with recurrences. For example, if we
start with an input of size $n$ and our recurrence halves the input size at each iteration,
\begin{equation*}
	T(n) =
	\begin{cases}
		c& \text{if }n=1\\
		T(n/2)& \text{otherwise}
	\end{cases}
\end{equation*}
Then we know that we will need at least $\lg n$ iterations to reach a constant case as,
\begin{equation*}
	T\left(\frac{n}{2\cdot2\cdots2}\right) = T\left(\frac{n}{2^{\lg n}}\right)\approx T(1)
\end{equation*}
Remember, we are simulating counting in a machine which starts from $0$ and hence the $\approx$ above instead
of $=$.
