\section{Getting Started}
\label{sec:chapter2}

\subsection{Insertion Sort - Exercises}

\subsubsection*{2.1--2}

\begin{codebox}
	\Procname{$\proc{Decreasing-Insertion-Sort}(A)$}
	\li \For $i = 1$ \To $A.\text{length}-1$
	\li	\Do
				key $= A[i]$
	\li 	$j = i - 1$
	\li		\While $j > -1$ \textbf{and} $A[j] <$ key
	\li		\Do
					$A[j+1] = A[j]$
	\li			$j = j - 1$
				\End
	\li		$A[j+1] =$ key
			\End
\end{codebox}

\subsubsection*{2.1--3}

\begin{codebox}
	\Procname{$\proc{Linear-Search}(A,\nu)$}
	\li \For $i = 0$ \To $A.\text{length}-1$
	\li \Do
				\If $A[i] \isequal \nu$
	\li		\Then
					\Return $i$
				\End
			\End
	\li		\Return \const{nil}
\end{codebox}

\begin{invariant}
	At the start of each iteration of the \textbf{for} loop (lines 1--4) $i-1$ is not an index of $A$ such that $A[i-1]=\nu$.
\end{invariant}

\begin{proof}
	Let us now prove the correctness of our algorithm. Suppose $i=0$, then $i-1$ is clearly not an index of $A$ and hence $A[i-1]$ is undefined. Now suppose the loop invariant is true for some $i$, that is, $i-1$ is not an index of $A$ such that $A[i-1]=\nu$, or equivalently, $A[i-1]\neq\nu$. Then at line 3 the \textbf{if} loop will \textbf{return} $i$ if $A[i]=\nu$, in which case the \textbf{for} loop terminates and there is no further iteration. Otherwise, if $A[i]\neq\nu$ then at the start of the next for loop iteration $(i+1)-1$ is not an index of $A$ such that $A[(i+1)-1]=\nu$. Finally, for termination to occur we have either $i=n+1$ where $n=A.\text{length}$ in which case the algorithm returns NIL indicating $\nu$ is not an element of $A$. Otherwise, termination occurs because of the nested \textbf{if} on line 3 which causes the algorithm to return $i$ which indicates the index of $A$ such that $A[i]=\nu$.
\end{proof}

\subsubsection*{2.1--4}

\begin{inp}
	Two sequences of $n$ integers, $A=(a_{1},\ldots,a_{n})$ and $B=(b_{1},\ldots,b_{n})$, such that $0\leq a_{i},b_{i}\leq 1$ for $i=1,\ldots,n$. Least significant digits are first.
\end{inp}

\begin{outp}
	An array $C=(c_{1},\ldots,c_{n},c_{n+1})$ such that $0\leq c_{i}\leq 1$ for $i=1,\ldots,n+1$ and $C'=A'+B'$ where $\cdot'$ is the integer represented by $\cdot$.
\end{outp}

\begin{codebox}
	\Procname{$\proc{Binary-Addition}(A,B)$}
	\li \Define integer $C[A.\text{length}+1]$
	\li overflow $=0$
	\li	\For $i = 0$ \To $A.\text{length}-1$
	\li \Do
				$C[i] = (A[i] + B[i] + \text{overflow})$ \% $2$
	\li		overflow $= (A[i] + B[i] + \text{overflow})/2$
			\End
	\li $C[i] =$ overflow
	\li \Return $C$
\end{codebox}

\subsection{Analysing Algorithms - Exercises}

\subsubsection*{2.2--1}

The function is $\Theta(n^{3})$

\subsubsection*{2.2--2}

\begin{multicols}{2}
	\begin{codebox}
		\Procname{$\proc{Selection-Sort}(A)$}
		\li	\For $i = 0$ \To $A.\text{length}-2$
		\li	\Do
					min $= i$
		\li		\For $j = i+1$ \To $A.$length$-1$
		\li		\Do
						\If $A[j] < A[\text{min}]$
		\li			\Then
							min = $j$
						\End
					\End
		\li $M = A[\text{min}]$
		\li $A[\text{min}] = A[i]$
		\li $A[i] = M$
					\End
		\end{codebox}
		\columnbreak
		\begin{tabular}{ l l }
			cost & times\\[4pt]
			$c_{1}$ & $n$\\
			$c_{2}$ & $n-1$\\
			$c_{3}$ & $\sum_{i=0}^{n}(n-i+1)$\\
			$c_{4}$ & $\sum_{i=0}^{n}(n-i)$\\
			$c_{5}$ & $\sum_{i=0}^{n}t_{i}$\\
			$c_{6}$ & $n-1$\\
			$c_{7}$ & $n-1$\\
			$c_{8}$ & $n-1$
		\end{tabular}
\end{multicols}

\begin{invariant}
	At the start of each iteration of the \textbf{for} loop (lines 1--8) the sub-array $A[0\ldots i]$ is sorted in non-decreasing order.
\end{invariant}

The algorithm only needs to run for the first $n-1$ elements since this will arrange the $n-1$ smallest elements in non-decreasing order, ensuring the $n^{\text{th}}$ element at the end is in the appropriate position. That is, $A[n]\geq A[i]$ for $i=0,\ldots,n-2$.

The best-case running time occurs when the given array is already sorted from smallest to largest. In such a case $t_{i}=0$ since we never need to re-assign the minimum index. The runtime equation is, 
\begin{equation*}
	\begin{aligned}
		T(n) &= c_{1}n + (c_{2}+c_{6}+c_{7}+c_{8})(n-1) + c_{3}\sum_{i=0}^{n}(n-i+1) + c_{4}\sum_{i=0}^{n}(n-i)\\
		&= c_{1}n + (c_{2}+c_{6}+c_{7}+c_{8})(n-1) + c_{3}\left((n+1) + \frac{n}{2}(n+1)\right) + c_{4}\left(n + \frac{n}{2}(n-1)\right)\\
		&= (c_{3}+c_{4})\frac{n^{2}}{2} + (c_{1} + c_{2} + c_{6} + c_{7} + c_{8} + \frac{3}{2}c_{3} + \frac{1}{2}c_{4})n + (c_{2}+c_{6}+c_{7}+c_{8} + c_{3})
	\end{aligned}
\end{equation*}
and so the best-case running time is $\Theta(n^{2})$. In a worst-case scenario, the array given to the procedure is in descending order, however this would only include an additional term to $T(n)$ above,
\begin{equation*}
	c_{5}\sum_{i=0}^{n}(n-1) = c_{5}\left(n + \frac{n}{2}(n-1)\right) = \frac{1}{2}c_{5}(n^{2}+n)
\end{equation*}
since here line 5 will re-assign the minimum for all remaining entries in the array. So the runtime in a worst-case scenario is also $\Theta(n^{2})$.

\subsubsection*{2.2--3}

\begin{multicols}{2}
	\begin{codebox}
		\Procname{$\proc{Linear-Search}(A,\nu)$}
		\li \For $i = 0$ \To $A.\text{length}-1$
		\li	\Do
					\If $A[i] \isequal \nu$
		\li		\Then
						\Return $i$
					\End
				\End
		\li \Return \const{nil}
	\end{codebox}
	\columnbreak
	\begin{tabular}{ l l }
		cost & times\\
		$c_{1}$ & $n+1$\\
		$c_{2}$ & $n$\\
		$c_{3}$ & $t_{1}$\\
		$c_{4}$ & $t_{2}$
	\end{tabular}
\end{multicols}

If each of the $n$ elements of $A$ have equal probability $p$ to be $\nu$ then the expected value is,
\begin{equation*}
	E[\nu] = 0\times\frac{1}{n}+1\times\frac{1}{n} + 2\times\frac{1}{n} + \cdots + n\times\frac{1}{n} = \frac{1}{n}\sum_{i=1}^{n}i = \frac{1}{n}\frac{n}{2}(n+1) = \frac{n+1}{2}
\end{equation*}
and hence on average we need to search through $\frac{n+1}{2}$ elements to find $\nu$. In the worst case we need to search $n$ elements since $\nu$ is not present in $A$. We have the following runtime equation,
\begin{equation*}
	T(n) = c_{1}(n+1) + c_{2}n + c_{3}t_{1} + c_{4}t_{2}
\end{equation*}
In the average-case $t_{1}=\frac{1}{2}=t_{2}$ then,
\begin{equation*}
	T(n) = (c_{1} + c_{2})n + c_{1} + \frac{1}{2}(c_{3} + c_{4})
\end{equation*}
and so the runtime is $\Theta(n)$. In the worst-case $t_{1}=0$ and $t_{2}=1$ so the runtime equation is,
\begin{equation*}
	T(n) = (c_{1} + c_{2})n + c_{1} + c_{3}
\end{equation*}
and so we still have $\Theta(n)$ runtime.

\subsubsection*{2.2--4}

Implement a checking loop/statement to return the procedure if in a best-case scenario. For example in Selection-Sort we can implement an initial loop that checks if the given array is already in sorted order and then return,
\begin{multicols}{2}
	\begin{codebox}
		\Procname{}
		\li \For $i = 0$ \To $A.\text{length}-2$
		\li \Do
					\If $A[i] > A[i+1]$
		\li		\Then
						\Break
					\End
				\End
		\li \If $i \isequal A.\text{length}-2$
		\li \Then
					\Return
				\End
	\end{codebox}
	\begin{tabular}{ l l }
		cost & times\\
		$c_{1}$ & $n$\\
		$c_{2}$ & $n-1$\\
		$c_{3}$ & $t_{1}$\\
		$c_{4}$ & $1$\\
		$c_{5}$ & $t_{2}$
	\end{tabular}
\end{multicols}
In such a case the runtime will be,
\begin{equation*}
	T(n) = (c_{1}+c_{2})n - c_{2} + c_{4} + c_{5}
\end{equation*}
which is $\Theta(n)$ a significant improvement over $\Theta(n^{2})$ in the above exercise.

\subsection{Designing Algorithms - Exercises}

\subsubsection*{2.3--1}

We first divide the array into sub-arrays until we have arrays of length $1$.
\begin{center}
	\small\Tree [.{$\left[3, 41, 52, 26, 38, 57, 9, 49\right]$} [.{$\left[3, 41, 52, 26\right]$} [.{$\left[3, 41\right]$} [{$\left[3\right]$} ] [{$\left[41\right]$} ] ] [.{$\left[52, 26\right]$} [{$\left[52\right]$} ] [{$\left[26\right]$} ] ] ] [.{$\left[38, 57, 9, 49\right]$} [.{$\left[38, 57\right]$} [{$\left[38\right]$} ] [{$\left[57\right]$} ] ] [.{$\left[9, 49\right]$} [{$\left[9\right]$} ] [{$\left[49\right]$} ] ] ] ]
\end{center}
Then we merge to eventually recover the original array in sorted order.
\begin{center}
	\small\Tree [.{$\left[3, 9, 26, 38, 41, 49, 52, 57\right]$} [.{$\left[3, 26, 41, 52\right]$} [.{$\left[3, 41\right]$} [{$\left[3\right]$} ] [{$\left[41\right]$} ] ] [.{$\left[26, 52\right]$} [{$\left[52\right]$} ] [{$\left[26\right]$} ] ] ] [.{$\left[9, 38, 49, 57\right]$} [.{$\left[38, 57\right]$} [{$\left[38\right]$} ] [{$\left[57\right]$} ] ] [.{$\left[9, 49\right]$} [{$\left[9\right]$} ] [{$\left[49\right]$} ] ] ] ]
\end{center}

\subsubsection*{2.3--2}

\begin{codebox}
	\Procname{$\proc{Merge}(A,p,q,r)$}
	\li $n_{1} = q - p$
	\li $n_{2} = r - q - 1$
	\li \Define integers $L[0\ldots n_{1}]$ and $R[0\ldots n_{2}]$
	\li \For $i = 0$ \To $n_{1}$
	\li \Do
				$L[i] = A[p+i]$
			\End
	\li \For $j = 0$ \To $n_{2}$
	\li \Do
				$R[j] = A[q+j+1]$
			\End
	\li $i = 0$
	\li $j = 0$
	\li \For $k = p$ \To $r$
	\li	\Do
				\If $i > n_{1}$
	\li		\Then
					$A[k] = R[j]$
	\li			$j = j + 1$
	\li 	\ElseIf $j > n_{2}$
	\li		\Then
					$A[k] = L[i]$
	\li			$i = i + 1$
	\li		\ElseIf $L[i] \leq R[j]$
	\li		\Then
					$A[k] = L[i]$
	\li			$i = i + 1$
	\li		\Else
	\li			$A[k] = R[j]$
	\li			$j = j + 1$
			\End
\end{codebox}

\subsubsection*{2.3--3}

\begin{proposition}
	\label{prop:power-2-recurrence}
	If $n=2^{k}$ for $k\in\mathbb{N}\setminus\{0\}$ then the solution of,
	\begin{equation*}
		T(n) =
		\begin{cases}
			2 &\text{if }k=1\\
			2T(n/2) + n &\text{if }k>1
		\end{cases}
	\end{equation*}
	is $T(n)=n\lg n$.
\end{proposition}

\begin{proof}
	If $k=1$ we have $n=2$ so $T(2) = 2 = 2\lg 2$. Now assume this is true for some $k=m>1$ then $T(2^{m}) = 2^{m}\lg 2^{m}$ so for $2^{m+1}$ we have the recurrence,
	\begin{equation*}
		\begin{aligned}
			T(2^{m+1}) = 2T(2^{m+1}/2) + 2^{m+1} &= 2T(2^{m}) + 2\cdot2^{m}\\
			&= 2\cdot2^{m}\lg 2^{m} + 2\cdot 2^{m}\\
			&= 2^{m+1}\left(\lg 2^{m} + 1\right)\\
			&= 2^{m+1}\left(\lg 2^{m} + \lg 2\right) = 2^{m+1}\lg 2^{m+1}
		\end{aligned}
	\end{equation*}
	Hence the solution for any $n=2^{k}$, $k\in\mathbb{N}\setminus\{0\}$, is $T(n)=n\lg n$.
\end{proof}

\subsubsection*{2.3--4}

Let $T(n)$ be the time to sort an array of length $n$ and $I(n)$ be the time to insert an element into an array of length $n$.
\begin{equation*}
	T(n) =
	\begin{cases}
		\Theta(1)& \text{if }n=1\\
		T(n-1) + I(n-1)& \text{otherwise}
	\end{cases}
\end{equation*}

\subsubsection*{2.3--5}

\begin{codebox}
	\Procname{$\proc{Binary-Search}(A, \nu)$}
	\li mid $= \lfloor A.\text{length}/2 \rfloor$
	\li \If $A[\text{mid}] \isequal \nu$
	\li \Then
				\Return mid
			\End
	\li \ElseIf mid $\isequal 0$
	\li	\Then
				\Return \const{false}
			\End
	\li \ElseIf $A[\text{mid}]<\nu$
	\li	\Then
				$\proc{Binary-Search}(A[\text{mid}\ldots A.\text{length}-1], \nu)$
			\End
	\li \Else
	\li	\Then
				$\proc{Binary-Search}(A[0\ldots \text{mid}], \nu)$
			\End
\end{codebox}

The running time of this algorithm consists of the re-running binary-search on arrays of approximate length $n/2$ twice and checking if the middle value of the total array is the target value, $\nu$. We can express this as the recurrence,
\begin{equation*}
	T(n) =
	\begin{cases}
		\Theta(1)& \text{if }n=1\\
		T(n/2) + c& \text{otherwise}
	\end{cases}
\end{equation*}

The worst case scenario occurs when $A$ does not contain $\nu$ in which case we can expand the recurrence as,
\begin{equation*}
	\begin{aligned}
		T(n) &= T\left(\frac{n}{2}\right) + c\\
		&= T\left(\frac{n}{2\cdot2}\right) + (1 + 1)c\\
		&= T\left(\frac{n}{2\cdot2\cdot2}\right) + (1 + 1 + 1)c\\
		&\vdots\\
		&= T\left(\frac{n}{2^{\lg n + 1}}\right) + (\lg n + 1)c \approx T(1) + (\lg n + 1)c
	\end{aligned}
\end{equation*}
So after $\lg n + 1$ iterations the algorithm returns $\const{nil}$ and from above we have that the runtime is $\Theta(\lg n)$.

\subsubsection*{2.3--6}

No, this is not possible since insertion-sort will still need to shift $i-1$ elements which will always create $\Theta(n^{2})$ time in the worst case.

\subsubsection*{2.3--7}

We first need to sort $S$ in ascending order. We can then determine whether there are two elements of $S$ that sum to $x$ by performing a binary-search for $x-S[i]$ in $S$.
\begin{codebox}
	\Procname{$\proc{Sum-Decomposition}(S,x)$}
	\li $S = \proc{Merge-Sort}(S,0,S.\text{length})$
	\li \For $i=0$ \To $S.\text{length}$
	\li \Do
				\If $\proc{Binary-Search}(S,x-S[i]) \neq \const{nil}$
	\li		\Then
					\Return \const{true}
				\End
			\End
	\li \Return \const{false}
\end{codebox}

From earlier in the chapter merge-sort will at worst take $\Theta(n\lg n)$ and from 2.3--5 binary-search is at worst $\Theta(\lg n)$. However as we loop on binary-search the worst-case for the loop on lines 2--4 will be $\Theta(n\lg n)$. Hence, sum-decomposition has a worst-case runtime of $\Theta(n\lg n)$.
